%%%%%%%%%%%%%%%%%%%%%%%%%%%%%%%%%%%%%%%%%
% Medium Length Professional CV
% LaTeX Template
%
% This template has been downloaded from:
% http://www.LaTeXTemplates.com
%
% Original author:
% Trey Hunner (http://www.treyhunner.com/)
%
% Important note:
% This template requires the resume.cls file to be in the same directory as the
% .tex file. The resume.cls file provides the resume style used for structuring the
% document.
%
%%%%%%%%%%%%%%%%%%%%%%%%%%%%%%%%%%%%%%%%%

%------------------------------------------------------------------------------
%	PACKAGES AND OTHER DOCUMENT CONFIGURATIONS
%------------------------------------------------------------------------------
\newcommand{\matlab}{{\sc Matlab }}
\newcommand{\matlabx}{{\sc Matlab}}
\newcommand{\mixten}{{\sc Mi\xten }}
\newcommand{\mixtenx}{{\sc Mi\xten}}
\newcommand{\xten}{{\sc X10 }}
\newcommand{\xtenx}{{\sc X10}}
\newcommand{\xtenb}{{\textbf{\textsc{X10}}}}
\newcommand{\discode}{{\sc Discode}}
\documentclass{resume} % Use the custom resume.cls style
\usepackage[hyphens]{url}
\usepackage{hyperref}
\usepackage{setspace} 

\usepackage[left=0.5in,top=0.25in,right=0.5in,bottom=0.2in, footskip=0.0in]{geometry} % Document margins
\usepackage{marginnote}

\name{\textsc{Vineet} \textsc{Kumar}} % Your name
\begin{document}
\begin{hSubsection}
{
    \href{https://github.com/vn7kr}{github.com/vntkg},
    \href{https://linkedin.com/in/vineet7kumar}{linkedin.com/in/vntkg}
}
{
	\textbf{vineet.kumar@mail.mcgill.ca, +34-653-60-21-02}
}
{%<Address>}
}
\end{hSubsection}



%------------------------------------------------------------------------------
%SUMMARY SECTION
%------------------------------------------------------------------------------
\smallskip \smallskip 
\begin{rSection}{} \smallskip 
\begin{lSubsection} 
\item \textbf{11 years} of relevant work experience
        $\vert$ \textbf{M.Sc Computer Science}, McGill University, Canada
\item Languages - C++, Python, Java $\vert$ Interests - Compilers, Computer Architecture, Accelerators, Supercomputing, LLVM, RISC-V
\end{lSubsection}
\end{rSection}

%------------------------------------------------------------------------------
%WORK EXPERIENCE SECTION
%------------------------------------------------------------------------------
\begin{rSection}{Work Experience}

\begin{rSubsection}{Barcelona Supercomputing Center}
	{April 2019 - Present}{Research Engineer}{Barcelona, Spain}
\item As part of the European Processor Initiative project, I am in charge of
	adding autovectorization support in the LLVM optimizer for the
	RISC-V V-extension based accelerator processor.
\item Collaborate with the LLVM community to bring predicated loop vectorization
  support to LLVM open source codebase. Participate in reviews and
  discussions for adding scalable vector support to LLVM.
\end{rSubsection}

\begin{rSubsection}{Huawei Canada Research Center - Compiler Technologies Lab}
	{September 2017 - May 2018}{Senior Software Engineer}{Toronto, ON, Canada}
\item Developed a new way to dynamically resize available heap size in the JVM (with Serial GC).
\item Contributed to the compiler framework being developed for Huawei's new AI processor.
\end{rSubsection}

\begin{rSubsection}{INRO}{December 2014 - September 2017}{Senior Developer}{Montreal,
        QC, Canada}
\item Wrote a \textbf{new compiler and memory management system} for Emme's
        matrix calculator language. Emme is a travel demand modelling system
        for transportation forecasting, used by some of the world's most
        populous cities.
\begin{lsubSubsection} 
\item Achieved \textbf{up to 30x faster} performance. Efficient even for
        computations on \textbf{large matrices (over 1 GB)}. (C++ and Python)
\end{lsubSubsection}
\item Designed and developed \textbf{data analytics tools for public transit
        data}.
\begin{lsubSubsection}
\item Enabled our clients to visually analyze and query things like
        loads, delays, and stop activities. (Python)
\end{lsubSubsection}
\item Designed and built the \textbf{data import backend and API} for CityPhi,
        an analytics platform for spatial and mobility data at scale.
\begin{lsubSubsection}
\item Support for various geographical and transit data formats like shapefile,
        OSM and GTFS.
\item Optimized to handle large datasets by importing data only in specified
        spatial and/or time windows. (C++ and Python)
\end{lsubSubsection}
\end{rSubsection}

\begin{rSubsection}{ISENCORE Technologies}{September 2013 - December 2014}{CTO
        and co-founder}{Montreal, QC, Canada}
\item Won \textbf{first prize with \$10,000 in funding} in the \textbf{Mcgill
        Dobson cup} (SME category) 2014 startup competition. 
\item Delivered the \textbf{winning pitch} to get selected as \textbf{one of
        the 20 startups worldwide} to present at SLUSH 2014.
\item Developed the \textbf{3D object discretization} module for
        Quirdity, ISENCORE's 3D simulation engine. 
\end{rSubsection}

\begin{rSubsection}{McGill University - Sable Compilers Research Lab}{January
        2012 - April 2014}{Research and Teaching}{Montreal, QC, Canada} 
\item \textbf{Research Assistant, Sable Lab} - My research included program
        analysis and static compilation of dynamic languages.

\begin{lsubSubsection}
\item Wrote \textbf{\mixtenx: a \matlab to \xten (programming language) compiler for
        high-performance}. (Java)(\href{http://bit.ly/getmix10}{\em{bit.ly/getmix10}})  
\item Achieved \textbf{7 times (mean) faster} performance compared to the
        standard \matlab implementation.
\item Designed a new algorithm to identify and safely typecast
        floating point values to integers at compile time for improved
        performance.
\item Discovered a \textbf{severe performance bottleneck} in the \xten compiler
        and helped improve the \xten compiler.  
\end{lsubSubsection}
\item \textbf{Teaching Assistant} - Program Analysis and Transformations,
	Compiler Design, and Introduction to Computer Systems.
\end{rSubsection}

\begin{rSubsection}{Infosys (for AT\&T)}{September 2008 - August
        2011}{Systems Engineer}{Pune, India}
\item \textbf{Led} a team of 4 for \textbf{performance management}
	of AT\&T's online and mobility applications. Test and resolve performance issues.
%\begin{lsubSubsection}
%\item My team's job was to design and develop tests, analyze results, and troubleshoot performance issues.  
%\end{lsubSubsection}
\end{rSubsection}

\begin{rSubsection}{Sun Microsystems}{January 2007 - May 2008}{Intern - Student
        Tech Lead, APAC region/Campus Ambassador}{Bangalore, India}
\item One of \textbf{only 5 Tech Leads worldwide}. Evangelized and taught a course on OpenSolaris.  
\end{rSubsection}
\end{rSection}

%------------------------------------------------------------------------------
%	PUBLICATIONS SECTION
%------------------------------------------------------------------------------

\begin{rSection}{Publications}
\smallskip
\begin{lSubsection}


\item Vineet Kumar and Laurie Hendren. \mixten: Compiling \matlab to \xten for
	High Performance. In Proceedings of the 2014 ACM International
	Conference on \textbf{Object Oriented Programming Systems Languages \&
	Applications (OOPSLA `14)}.(\href{http://bit.ly/1papr1}{\em{bit.ly/1papr1}})

%\item \emph{Talk:} Vineet Kumar and Laurie Hendren. \mixten: Compiling \matlab
%for high performance computing via \xten. 12\textsuperscript{th}
%\textbf{Compiler-Driven Performance Workshop} at \textbf{CASCON
%2013}.(\href{http://webdocs.cs.ualberta.ca/~amaral/cascon/CDP13/#VinetKumar}{\em{bit.ly/1hXms8N}}) 
%
\item Vineet Kumar and Laurie Hendren. First steps to compiling \matlab to
	\xten. In Proceedings of the 2013 ACM SIGPLAN X10 Workshop, \textbf{X10
	`13} co-located with \textbf{PLDI
	2013}.(\href{http://bit.ly/2papr2}{\em{bit.ly/2papr2}})
\end{lSubsection}
\end{rSection}

%-----------------------------------------------------------------------------
%	EDUCATION SECTION
%-----------------------------------------------------------------------------
 
\begin{rSection}{Education}

\begin{rSubsection}{McGill University}{April 2014}{M.Sc. in
Computer Science}{Montreal, QC, Canada} 
%\item Master's thesis reviewed as \textbf{``Excellent''} by the external
%reviewer.
\item Won the DFW scholarship awarded to \textbf{exceptional international
        Research Master's} students.
\end{rSubsection}

\begin{rSubsection}{SASTRA University}{June 2008}{B.Tech. in Computer Science
\& Engineering}{Thanjavur, India} 
\item Won the {Dean's list scholarship} for being among the \textbf{top
10\%} students in the University. 
%\item \textbf{Co-founded and led} GLOSS(GNU Linux \& Open Source at SASTRA)
%club of the University.  
\end{rSubsection}

\end{rSection}

\end{document}
